% !TeX spellcheck = es_ES
% !TeX program = xelatex
\documentclass[12pt,letterpaper,twoside]{book} 
\usepackage{titletoc}       
\usepackage{ucs}                
\usepackage[spanish,es-nodecimaldot]{babel} 
\usepackage{fontspec}       
\usepackage[T1]{fontenc}       
\setsansfont{URWClassico}[
    Path=./URWClassico/,
    Extension = .ttf,
    UprightFont=*-Regular,
    BoldFont=*-Bold,
    ItalicFont=*-Italic,
    BoldItalicFont=*-BoldItalic
    ]   
\usepackage{color}
\usepackage{xcolor,colortbl}
    \definecolor{orangy}{RGB}{255,122,4}
    \definecolor{blueish}{RGB}{4,122,255}
    \definecolor{aqua}{RGB}{0,100,150}
    \definecolor{deepblue}{RGB}{0,102,205}
    \definecolor{yellowish}{RGB}{255,255,0}
    \definecolor{chapopcolor}{RGB}{0,100,150}
\usepackage{wallpaper} 
\usepackage[skip=0.4em,indent=1em]{parskip}
    \renewcommand{\baselinestretch}{1.1}   
    \raggedbottom       
\usepackage{graphicx}
    \graphicspath{ {images/} }
\usepackage{tikz}    
    \usetikzlibrary{babel}
    \usetikzlibrary{arrows.meta,calc,positioning,shapes,arrows,decorations.pathmorphing}
    \usetikzlibrary{cd}
\usepackage[left=4cm,right=4cm,top=2.7cm,bottom=2.5cm,outer=3cm,inner=3cm,heightrounded]{geometry}  
\usepackage[breaklinks=true,
colorlinks=true,linkcolor=blue!22!black,urlcolor=blue,citecolor=black,anchorcolor=blue!70!black,
bookmarks=true,pdfpagelayout=TwoPageRight]{hyperref}
\usepackage{tcolorbox}       
\usepackage{varwidth}        
    \tcbuselibrary{breakable,skins,theorems}



\begin{document}
\frontmatter
\begin{titlepage}
\newgeometry{left=7.5cm,top=1.2cm,right=1cm}
\begin{tikzpicture}[remember picture, overlay]
\node[opacity=1,inner sep=0pt] at (current page.center)
{\includegraphics[width=\paperwidth,height=\paperheight]{wall/wall45b}};
\end{tikzpicture}
\vphantom{A}
\vfill
\noindent{\fontsize{70}{48} \selectfont \bfseries\sffamily lógica básica.}
\\{\fontsize{20}{48} \selectfont \bfseries\sffamily una introducción al arte del razonamiento}
\par
\noindent
\makebox[0pt][l]{\rule{1.3\textwidth}{1pt}}
\par
\noindent
{\large\sffamily \href{mailto:}{\bfseries carlos romero}}
\par
\vskip\baselineskip
\noindent \textsf{(versión parcial de \today.)}
\end{titlepage}
\restoregeometry
\pagecolor{white}
\pagestyle{plain}
\end{document}
